\documentclass[a4paper, 12pt]{article}

\usepackage{fontspec}
\usepackage[ukrainian]{babel}

\usepackage{Alegreya} 
\setmonofont{0xProto-Regular} 

\usepackage{geometry}
\geometry{
    left=25mm,
    right=15mm,
    top=20mm,
    bottom=20mm
}

\usepackage{indentfirst}
\setlength{\parindent}{1.25cm}

\usepackage{setspace}
\singlespacing

\usepackage{amsmath, amsfonts, amssymb}
\usepackage{graphicx}
\usepackage{float}
\usepackage{hyperref}
\hypersetup{
    colorlinks=true,
    linkcolor=black,
    filecolor=magenta,      
    urlcolor=blue,
}

\usepackage{listings}
\usepackage{xcolor}

\renewcommand{\thesection}{\Roman{section}} 
\renewcommand{\thesubsection}{\thesection.\Roman{subsection}}

\definecolor{codegreen}{rgb}{0,0.6,0}
\definecolor{codegray}{rgb}{0.5,0.5,0.5}
\definecolor{codepurple}{rgb}{0.58,0,0.82}
\definecolor{backcolour}{rgb}{0.96,0.96,0.96}

\lstdefinestyle{mystyle}{
    backgroundcolor=\color{backcolour},   
    commentstyle=\color{codegreen},
    keywordstyle=\color{magenta},
    numberstyle=\tiny\color{codegray},
    stringstyle=\color{codepurple},
    basicstyle=\ttfamily\footnotesize,
    breakatwhitespace=false,         
    breaklines=true,                 
    captionpos=b,                    
    keepspaces=true,                 
    numbers=left,                    
    numbersep=5pt,                  
    showspaces=false,                
    showstringspaces=false,
    showtabs=false,                  
    tabsize=4,
    language=Python,
    frame=single
}

\lstset{style=mystyle}

\begin{document}
\directlua{token.set_macro(("studentname"),   os.getenv("REPORT_STUDENT_NAME")   or ("Student"),   ("global"))}
\directlua{token.set_macro(("studentcourse"), os.getenv("REPORT_STUDENT_COURSE") or ("Course"),    ("global"))}
\directlua{token.set_macro(("studentgroup"),  os.getenv("REPORT_STUDENT_GROUP")  or ("Group"),     ("global"))}
\directlua{token.set_macro(("professorname"), os.getenv("REPORT_PROFESSOR_NAME") or ("Professor"), ("global"))}

\begin{titlepage}
    \begin{center}
		{\Large
			МІНІСТЕРСТВО ОСВІТИ ТА НАУКИ УКРАЇНИ

			НАЦІОНАЛЬНИЙ ТЕХНІЧНИЙ УНІВЕРСИТЕТ УКРАЇНИ
			«КИЇВСЬКИЙ ПОЛІТЕХНІЧНИЙ ІНСТИТУТ ІМЕНІ ІГОРЯ СІКОРСЬКОГО»

			ФАКУЛЬТЕТ ІНФОРМАТИКИ ТА ОБЧИСЛЮВАЛЬНОЇ ТЕХНІКИ

			КАФЕДРА ІНФОРМАЦІЙНИХ СИСТЕМ ТА ТЕХНОЛОГІЙ
		}


		\vspace{60mm}
		{\large
			\textbf{ЗВІТ}

			\vspace{5mm}

			\textbf{До лабораторної роботи \textnumero 2}

			\vspace{5mm}

			з дисципліни «Технології Computer Vision»
            
            \vspace{2mm}
            
            (Сертифікатна програма Data Science із Sigma Software)
		}

	\end{center}

	\vfill

	\begin{minipage}[t]{0.30\textwidth}
		\textbf{Перевірив:}

        \professorname
	\end{minipage}
	\hfill
	\begin{minipage}[t]{0.30\textwidth}
		\textbf{Виконав:}

		студент групи \studentgroup

        \studentname
	\end{minipage}

	\vfill

	\begin{center}
		{\bf
			Київ

			КПІ Ім. Ігоря Сікорського

			2025
		}
	\end{center}
\end{titlepage}

\section{Мета і завдання лабораторної роботи}

\textbf{Мета роботи:} дослідити принципи та особливості практичного
застосування технологій обробки растрових та векторних цифрових зображень для
задач Computer Vision з використанням спеціалізованих програмних бібліотек.

\textbf{Завдання:}
\begin{enumerate}
    \item За даними від ArcGIS Living Atlas та Bing Maps підрахувати кількість
        будівель в головному кампусі КПІ.
    \item Порівняти отримані результати.
    \item Знайти комбінацію операцій обробки растрових зображень та
        векторизації, коли кількість знайдених будівель за ArcGIS буде
        відповідати реальній кількості (або еталону з Bing Maps).
\end{enumerate}

\section{Синтезована математична модель}

\subsection{Колірна модель HSV}
Для сегментації дахів будівель використано колірний простір \textbf{HSV} (Hue,
Saturation, Value). На відміну від RGB, HSV дозволяє відокремити інформацію про
колір (H) від його інтенсивності (V), що робить алгоритм більш стійким до тіней
та нерівномірного освітлення.

Сегментація виконується шляхом бінаризації за пороговими значеннями:
\[
    M(x,y) = \begin{cases} 
    1, & \text{якщо } H \in [H_{min}, H_{max}] \land S \in [S_{min}, S_{max}] \land V \in [V_{min}, V_{max}] \\
    0, & \text{інакше}
    \end{cases}
\]

\subsection{Морфологічні операції}
Для покращення бінарної маски використовуються морфологічні операції:
\begin{itemize}
    \item \textbf{Opening (Розмикання):} Ерозія з подальшою дилатацією ($A
        \circ B = (A \ominus B) \oplus B$). Використовується для видалення
        дрібного шуму.
    \item \textbf{Closing (Замикання):} Дилатація з подальшою ерозією ($A
        \bullet B = (A \oplus B) \ominus B$). Використовується для "заливання"
        дірок всередині об'єктів та об'єднання розірваних частин будівель.
\end{itemize}

\section{Архітектура та реалізація}

Програма складається з модуля обробки зображень \texttt{vision.py} та
інтерактивного ноутбука \texttt{task3.py}.

\subsection{Інтерактивний інтерфейс}
Для підбору параметрів використано бібліотеку \textbf{Marimo}, що дозволяє
динамічно змінювати порогові значення HSV, розмір ядра розмиття та параметри
морфології.

\section{Результати роботи}

\begin{figure}[H]
    \centering
    \includegraphics[width=1.0\textwidth]{media/arcgis_result.png}
\end{figure}

\subsection{Програмний код}

\lstinputlisting[language=Python, caption=vision.py]{../../src/lab2/vision.py}

\section{Висновки}
У ході лабораторної роботи було реалізовано пайплайн для обробки супутникових знімків:
\begin{enumerate}
    \item Використання колірного простору HSV дозволило ефективно виділити дахи
        будівель, ігноруючи асфальт та рослинність.
    \item Морфологічна операція \texttt{Closing} виявилася критично важливою
        для об'єднання окремих пікселів даху в єдиний контур будівлі, особливо
        на зображеннях низької якості.
    \item Фільтрація за площею (contour area) дозволила відсіяти дрібний шум
        (автомобілі, кіоски) та занадто великі об'єкти (помилки сегментації
        фону).
    \item Експериментально доведено, що для різних джерел даних (Bing vs
        ArcGIS) необхідні різні набори параметрів фільтрації.
\end{enumerate}

Завдання виконав \studentname

\end{document}
