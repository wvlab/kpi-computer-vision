\documentclass[a4paper, 12pt]{article}

\usepackage{fontspec}
\usepackage[ukrainian]{babel}

\usepackage{Alegreya} 

\setmonofont{0xProto-Regular} 

\usepackage{geometry}
\geometry{
    left=25mm,
    right=15mm,
    top=20mm,
    bottom=20mm
}

\usepackage{indentfirst}
\setlength{\parindent}{1.25cm}

\usepackage{setspace}
\singlespacing

\usepackage{amsmath, amsfonts, amssymb}
\usepackage{graphicx}
\usepackage{float}
\usepackage{hyperref}
\hypersetup{
    colorlinks=true,
    linkcolor=black,
    filecolor=magenta,      
    urlcolor=blue,
}

\usepackage{listings}
\usepackage{xcolor}

\renewcommand{\thesection}{\Roman{section}} 
\renewcommand{\thesubsection}{\thesection.\Roman{subsection}}

\definecolor{codegreen}{rgb}{0,0.6,0}
\definecolor{codegray}{rgb}{0.5,0.5,0.5}
\definecolor{codepurple}{rgb}{0.58,0,0.82}
\definecolor{backcolour}{rgb}{0.96,0.96,0.96}

\lstdefinestyle{mystyle}{
    backgroundcolor=\color{backcolour},   
    commentstyle=\color{codegreen},
    keywordstyle=\color{magenta},
    numberstyle=\tiny\color{codegray},
    stringstyle=\color{codepurple},
    basicstyle=\ttfamily\footnotesize,
    breakatwhitespace=false,         
    breaklines=true,                 
    captionpos=b,                    
    keepspaces=true,                 
    numbers=left,                    
    numbersep=5pt,                  
    showspaces=false,                
    showstringspaces=false,
    showtabs=false,                  
    tabsize=4,
    language=Python,
    frame=single
}

\lstset{style=mystyle}

\begin{document}
\directlua{token.set_macro(("studentname"),   os.getenv("REPORT_STUDENT_NAME")   or ("Student"),   ("global"))}
\directlua{token.set_macro(("studentcourse"), os.getenv("REPORT_STUDENT_COURSE") or ("Course"),    ("global"))}
\directlua{token.set_macro(("studentgroup"),  os.getenv("REPORT_STUDENT_GROUP")  or ("Group"),     ("global"))}
\directlua{token.set_macro(("professorname"), os.getenv("REPORT_PROFESSOR_NAME") or ("Professor"), ("global"))}

\begin{titlepage}
    \begin{center}
		{\Large
			МІНІСТЕРСТВО ОСВІТИ ТА НАУКИ УКРАЇНИ

			НАЦІОНАЛЬНИЙ ТЕХНІЧНИЙ УНІВЕРСИТЕТ УКРАЇНИ
			«КИЇВСЬКИЙ ПОЛІТЕХНІЧНИЙ ІНСТИТУТ ІМЕНІ ІГОРЯ СІКОРСЬКОГО»

			ФАКУЛЬТЕТ ІНФОРМАТИКИ ТА ОБЧИСЛЮВАЛЬНОЇ ТЕХНІКИ

			КАФЕДРА ІНФОРМАЦІЙНИХ СИСТЕМ ТА ТЕХНОЛОГІЙ
		}


		\vspace{60mm}
		{\large
			\textbf{ЗВІТ}

			\vspace{5mm}

			\textbf{До лабораторної роботи \textnumero 1}

			\vspace{5mm}

			з дисципліни «Технології Computer Vision»
		}

	\end{center}

	\vfill

	\begin{minipage}[t]{0.30\textwidth}
		\textbf{Перевірив:}

        \professorname
	\end{minipage}
	\hfill
	\begin{minipage}[t]{0.30\textwidth}
		\textbf{Виконав:}

		студент групи \studentgroup

        \studentname
	\end{minipage}

	\vfill

	\begin{center}
		{\bf
			Київ

			КПІ Ім. Ігоря Сікорського

			2025
		}
	\end{center}
\end{titlepage}

\section{Мета роботи}
Дослідити та програмно реалізувати математичні моделі афінних перетворень
графічних об'єктів (2D та 3D) з використанням матричних операцій та однорідних
координат. Опанувати створення динамічних сцен за допомогою Python та бібліотек
Marimo/Matplotlib.

\section{Завдання (Варіант 14)}

\subsection{Завдання І рівня (2D)}
Здійснити синтез математичних моделей та розробити скрипт для 2D перетворень:
\begin{itemize}
    \item \textbf{Фігура:} Шестикутник.
    \item \textbf{Операції:} Обертання – Переміщення – Масштабування.
    \item \textbf{Умови:} Циклічна реалізація, прихована траєкторія руху, виконання в межах графічного вікна.
\end{itemize}

\subsection{Завдання ІІ рівня (3D)}
Здійснити синтез моделей для 3D перетворень:
\begin{itemize}
    \item \textbf{Фігура:} Піраміда з трикутною основою.
    \item \textbf{Проекція:} Аксонометрична.
    \item \textbf{Динаміка:} Циклічне обертання навколо внутрішньої вісі, ефекти появи/зникнення (fade in/out), випадкова зміна положення та кольору після кожного циклу.
\end{itemize}

\section{Теоретичні відомості та математичні моделі}

Для реалізації перетворень використано \textbf{однорідні координати}
(Homogeneous Coordinates). Це дозволяє записати афінні перетворення (включаючи
паралельне перенесення) як лінійні операції множення матриць.

\subsection{2D простір}
Вектор координат точки: $P = [x, y, 1]^T$.
Матриця перетворення має розмірність $3 \times 3$.

Матриця масштабування ($S$), обертання ($R$) та переміщення ($T$):
\[
    S(s_x, s_y) = \begin{bmatrix}
        s_x & 0   & 0 \\
        0   & s_y & 0 \\
        0   & 0   & 1
    \end{bmatrix}, \quad
    R(\alpha) = \begin{bmatrix}
        \cos\alpha & -\sin\alpha & 0 \\
        \sin\alpha &  \cos\alpha & 0 \\
        0          & 0           & 1
    \end{bmatrix}, \quad
    T(d_x, d_y) = \begin{bmatrix}
        1 & 0 & d_x \\
        0 & 1 & d_y \\
        0 & 0 & 1
    \end{bmatrix}
\]

\subsection{3D простір та проекція}
Вектор координат: $P = [x, y, z, 1]^T$. Матриця перетворення має розмірність $4 \times 4$.

Для відображення 3D об'єкта на 2D екрані використано \textbf{аксонометричну
проекцію}, яка формується як добуток матриці ортогонального проектування
($P_{ort}$) та матриць повороту навколо осей $X$ та $Y$:
\[
    M_{proj} = P_{ort} \cdot R_y(\beta) \cdot R_x(\alpha)
\]

\section{Програмна реалізація}

Програма реалізована мовою Python з використанням середовища \textbf{Marimo}
для інтерактивності. Код розділено на бібліотеку трансформацій та виконувані
ноутбуки.

\subsection{Бібліотека трансформацій (src/lab1/transform.py)}
Реалізація генераторів матриць та функції композиції.

\lstinputlisting[
    language=python,
    caption=Функції генерації матриць та композиції
]{../src/lab1/transform.py}

\subsection{Генерація фігур (src/lab1/creation.py)}

\lstinputlisting[
    language=python,
    caption=Функції генерації матриць та композиції
]{../src/lab1/creation.py}

\section{Результати роботи}

\subsection{2D Анімація (Шестикутник)}
Реалізовано рух по колу, пульсацію та власне обертання.

\begin{figure}[H]
    \centering
    \includegraphics[width=0.8\textwidth]{media/task1_result.png}
    \caption{Результат роботи Task 1: 2D перетворення шестикутника.}
    \label{fig:task1}
\end{figure}

\newpage

\subsection{3D Анімація (Піраміда)}
Фігура з'являється, обертається та зникає. Кожен цикл відбувається в новому випадковому місці простору.

\begin{figure}[H]
    \centering
    \includegraphics[width=0.8\textwidth]{media/task2_result.png}
    \caption{Результат роботи Task 2: 3D піраміда в аксонометрії.}
    \label{fig:task2}
\end{figure}

\section{Висновки}

У ході лабораторної роботи було:
\begin{enumerate}
    \item Розроблено бібліотеку базових геометричних перетворень з
        використанням бібліотеки \texttt{numpy}.
    \item Використано розширені матриці координат ($3 \times N$ та $4 \times
        N$) для уніфікації операцій переміщення, масштабування та обертання.
    \item Реалізовано складну динаміку 3D-об'єкта: циклічна зміна прозорості
        (fade-in/fade-out) та телепортація об'єкта у випадкові координати.
\end{enumerate}

Завдання виконав \studentname

\end{document}
